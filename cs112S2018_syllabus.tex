\documentclass[11pt]{article}

% Edit this customize for an instructor

\newcommand{\instructorpronoun}[1]{his}

% Use this when displaying a new command

\newcommand{\command}[1]{``\lstinline{#1}''}
\newcommand{\program}[1]{\lstinline{#1}}
\newcommand{\url}[1]{\lstinline{#1}}
\newcommand{\channel}[1]{\lstinline{#1}}
\newcommand{\option}[1]{``{#1}''}
\newcommand{\step}[1]{``{#1}''}

\usepackage{pifont}
\newcommand{\checkmark}{\ding{51}}
\newcommand{\naughtmark}{\ding{55}}

\usepackage{listings}
\lstset{
  basicstyle=\small\ttfamily,
  columns=flexible,
  breaklines=true
}

% Define the headers and footers

\usepackage{fancyhdr}

\usepackage[margin=1in]{geometry}
\usepackage{fancyhdr}

\pagestyle{fancy}

\fancyhf{}
\rhead{Computer Science 112}
\lhead{Syllabus}
\rfoot{Page \thepage}
\lfoot{Spring 2018}

% Use elastic spacing around the headers

\usepackage{titlesec}
\titlespacing\section{0pt}{6pt plus 4pt minus 2pt}{4pt plus 2pt minus 2pt}

\newcommand{\syllabustitle}[1]
{
  \begin{center}
    \begin{center}
      \bf
      CMPSC 112\\Introduction to Computer Science II\\
      Spring 2018\\
      \medskip
    \end{center}
    \bf
    #1
  \end{center}
}

\begin{document}

\thispagestyle{empty}

\syllabustitle{Syllabus}

\vspace*{-.1in}
\subsection*{Course Instructor}
Dr.\ Gregory M.\ Kapfhammer\\
\noindent Office Location: Alden Hall 108 \\
\noindent Office Phone: +1 814--332--2880 \\
\noindent Email: \url{gkapfham@allegheny.edu} \\
\noindent Twitter: \url{@GregKapfhammer} \\
\noindent Web Site: \url{http://www.cs.allegheny.edu/sites/gkapfham/}

\subsection*{Instructor's Office Hours}

\begin{itemize}
  \itemsep0em

  \item Monday: 10:00 am--11:00 am (15 minute time slots)

  \item Tuesday: 4:30 pm--5:00 pm (15 minute time slots)

  \item Wednesday: 10:00 am--11:00 am and 2:30 pm--4:00 pm (15 minute time slots)

  \item Friday: 10:00 am--11:00 am and 2:30 pm--3:30 pm (15 minute time slots)

\end{itemize}

\vspace*{-.1in}

\noindent To schedule a meeting with me during my office hours, please visit my web site and click the ``Schedule'' link
in the top right-hand corner. Now, you can browse my office hours or schedule an appointment by clicking the correct
link and then reserving an open time slot. Students are also encouraged to post appropriate questions to a channel in
Slack, which is available at \url{https://CMPSC112Spring2018.slack.com/}, and monitored by the instructor and the
teaching assistants.

\subsection*{Course Meeting Schedule}

Lecture, Discussion, and Group Work Session: Monday and Wednesday 1:30 pm--2:20 pm \\
Practical Session: Friday 1:30 pm--2:20 pm \\
Laboratory Session: Tuesday, 2:30 pm--4:20 pm \\
Final Examination: Friday, May 4, 2018 at 7:00 pm

\subsection*{Course Description}

\begin{quote}

A continuation of CMPSC 111 with an emphasis on implementing, using, and
evaluating the computational structures needed to efficiently store and
retrieve digital data. Participating in hands-on activities that often require
teamwork, students create data structures and algorithms whose correctness and
performance they study through proofs and experimentation. Students continue to
refine their ability to organize and document a program's source code so that it
effectively communicates with the intended users and maintainers. During a
weekly laboratory session, students use state-of-the-art technology to complete
projects, reporting on their results through both written reports and oral
presentations. Prerequisite: CMPSC 111 or permission of the instructor.
Distribution Requirements: QR, SP.\@ \\

More details about this course are available at its mobile-ready web site:
\url{http://www.cs.allegheny.edu/sites/gkapfham/teaching/cs112S2018/}.

\end{quote}

\subsection*{Course Objectives}

The process of implementing and evaluating correct and efficient software involves the application of many interesting
theories, techniques, and tools. In addition to learning problem solving and computational thinking skills, students
will learn how to design, implement, test, and analyze algorithms and data structures in an object-oriented programming
language. Students will learn more about fundamental concepts such as recursion, searching, and sorting while also
discovering how to analytically and empirically evaluate the performance of computer programs. Students will also
investigate the use, implementation, and testing of data structures such as stacks, queues, and lists. Along with
learning more about how to work in a team of diverse software developers, students will enhance their ability to write
and present ideas about software in a clear, concise, and compelling fashion. Students will also develop a richer
understanding of the fascinating connections between computer science and other disciplines in the social and natural
sciences and the humanities.

\subsection*{Performance Objectives}

At the completion of this semester, a student should be comfortable with the object-oriented programming paradigm. Also,
students should be able to handle many of the important, yet accidental, aspects of implementing programs in the Java
programming language. That is, students should be able to use text editors, build systems, and integrated development
environments and understand both the purpose and use of shell environment variables such as the {\tt CLASSPATH}.
Students should have a toolkit of data structures that they can use to respond to the challenges that they encounter
during the development and analysis of software. Students must have a strong grasp of the basic components of an
object-oriented programming language and an ever-deepening knowledge of topics like recursion, searching, and sorting.
Along with demonstrating the ability to use both in-person discussions and software tools to collaborate with a group of
diverse team members, students should have a knowledge of the analytical and empirical techniques used to measure
performance.

\subsection*{Required Textbook}

\noindent{\em Data Structures and Algorithms in Java}. Michael T.\ Goodrich, Roberto Tamassia, and Michael H.\
Goldwasser. Sixth Edition, ISBN--10: 1118771338, ISBN--13: 978--1118771334, 720 pages, 2014. \\ (References to the
textbook are abbreviated as ``DSAAIJ'' on the course Web site).

\noindent
Students who want to improve their technical writing skills may consult the following books.

\noindent{\em BUGS in Writing: A Guide to Debugging Your Prose}. Lyn Dupr\'e. Second Edition,  ISBN--10: 020137921X,
ISBN-13: 978--0201379211, 704 pages, 1998.

\noindent{\em Writing for Computer Science}. Justin Zobel. Second Edition,  ISBN--10: 1852338024, ISBN--13:
978--1852338022, 270 pages, 2004.

\noindent Along with reading the required textbook, you may be asked to study additional articles from a wide variety of
conference proceedings, scientific journals, and the popular press.

\subsection*{Course Policies}

\subsubsection*{Grading}

The grade that a student receives in this class will be based on the following categories. All of these percentages are
approximate and, if the need to do so presents itself, it is possible for the course instructor to change the assigned
percentages during the academic semester.

\begin{center}
  \begin{tabular}{c c}
  \begin{tabular}{ll}
    Class Participation        & 5\%  \\
    SEED Project Participation & 5\%  \\
    Midterm Examination        & 15\% \\
    Final Examination          & 15\% \\
  \end{tabular} &
  \begin{tabular}{ll}
    Mastery Quizzes            & 10\% \\
    Laboratory Assignments     & 30\% \\
    Practical Assignments      & 10\% \\
    Final Project              & 10\%
  \end{tabular}
  \end{tabular}
\end{center}

\vspace*{-.05in}

\noindent
These grading categories have the following definitions:

\vspace*{-.05in}

\begin{itemize}

  \item {\em Class Participation\/}: All students are required to actively participate during all of the course
    sessions. Your participation will take forms such as answering questions about the reading assignments, asking
    constructive questions of group members, giving presentations, and leading a discussion. You also must regularly
    participate in the discussions in the course's Slack team. A student may request feedback on and will receive a
    final grade for this category.

  \item {\em SEED Project Participation\/}: I will regularly publish blog posts, available at
    \url{http://www.cs.allegheny.edu/sites/gkapfham/blog/}, in which I interview a software developer in industry. When
    a new interview is available, every student should read it and then tweet a response to points raised by the
    interviewee. Along with linking to the web site of the blog post, mentioning \url{@GregKapfhammer}, and using the
    hashtag \url{#SEED}, your tweet should share what you learned from the interviewee. You may also connect with the
    interviewee through LinkedIn.

  \item {\em Mastery Quizzes\/}: The two quizzes will cover all of the material in their associated module(s). While the
    second quiz is not cumulative, it will assume that a student has a basic understanding of the material that was the
    focus of the first quiz. The date for each of the quizzes will be announced at least one week in advance of the
    scheduled date. Unless prior arrangements are made with the course instructor, all students will be expected to take
    these two quizzes on the scheduled date and to complete the quizzes in the stated period of time.

  \item {\em Midterm Examination\/}: The midterm is an hour-long cumulative test covering all of the material from the
    class, practical, and laboratory sessions, as outlined on the review sheet. Unless prior arrangements are made with
    the course instructor, all students will be expected to take this test on the scheduled date and complete the test
    in the stated period of time.

  \item {\em Final Examination\/}: The final examination is a three-hour cumulative test. By enrolling in this course,
    students agree that, unless there are severe extenuating circumstances, they will take the final examination at the
    date and time stated on the first page of the syllabus.

  \item {\em Laboratory Assignments\/}: These assignments invite students to explore different techniques for designing,
    implementing, evaluating, and documenting software solutions to challenging problems that often have a connection to
    real-world concerns. Many of the assignments will require students to conduct experiments and collect, analyze, and
    write about data sets. To best ensure that students are ready to develop software in both other classes at Allegheny
    and after graduation, students will complete assignments both on an individual basis and in teams. When teamwork is
    required, the instructor will often assign individuals to teams.

  \item {\em Practical Assignments\/}: Graded on a credit/no-credit basis, these assignments allow students to practice
    the technical skills that they learned in previous class and laboratory sessions.

  \item {\em Final Project\/}: This project will furnish you with the description of a problem and ask you to design,
    implement, describe, and orally present a correct and carefully evaluated solution. Completion of the final project
    will require you to apply all of the knowledge and skills that you have acquired during the semester to solve a
    technical problem and, whenever possible, make your solution and results publicly available in a free and open
    fashion.

\end{itemize}

\subsubsection*{Assignment Submission and Evaluation}

All assignments will have a stated due date. Electronic versions of the laboratory, practical, and final project
assignments must be submitted to a student's GitHub repository; students will learn how to use version control with
GitHub during the first laboratory and practical sessions. No credit will be awarded for any course work that is not
submitted to your GitHub repository with the required name. Unless specified otherwise, all assignments must be
turned in at the beginning of the session that is one week after the day the assignment was released. Unless special
arrangements are made with the course instructor, no work will be accepted after the published deadline.

Using a report that the instructor shares with you through the commit log in GitHub, you will privately received a grade
for and feedback on each assignment. Your grade will be a function of whether or you not completed correct work and
submitted it by the deadline. Other factors (e.g., the quality of your source code and technical writing) will also
influence your assignment's grade.

\subsubsection*{Course Attendance}

It is mandatory for all students to attend all of the class, practical, and laboratory sessions. If, due to extenuating
circumstances, you will not be able to attend a session, then, whenever possible, please see the course instructor at
least one week in advance to describe your situation. Students who miss more than five unexcused sessions will have
their final grade in the course reduced by one letter grade. Students who miss more than ten of the aforementioned
events will fail the course.

\subsubsection*{Use of Laboratory Facilities}

Throughout the semester, we will employ many different software tools that computer scientists use during the design,
implementation, and evaluation of computer software. Since it is unlikely that an unmodified computer will have the
software that correctly supports the completion of the laboratory assignments, unless there are extenuating
circumstances, students struggling with computer setup must complete all of their work for this course while using the
department's laboratory facilities.

\subsubsection*{Class Preparation}

In order to minimize confusion and maximize learning, students must invest time to prepare for the class discussions,
lectures, and practical sessions. During the class periods, the course instructor will often pose challenging questions
that could require group discussion, the creation of a program or data set, a vote on a thought-provoking issue, or a
group presentation. Only students who have prepared for class by reading the assigned material and reviewing the
current laboratory and practical assignments will be able to effectively participate in these discussions.

More importantly, only prepared students will be able to acquire the knowledge and skills that they need to be
successful in this course, subsequent courses, and the field of computer science. In order to help students remain
organized and to effectively prepare for classes, the course instructor will maintain a class schedule with reading
assignments and presentation slides. During the class sessions students will also be required to download, use, and
modify programs and data sets that are made available through means such as the course web site and a GitHub repository.

\subsubsection*{Seeking Assistance}

Students who are struggling to understand the knowledge and skills developed in a class, laboratory, or practical
session are encouraged to seek assistance from the course instructor, the teaching assistants, or the departmental
tutors. Throughout the semester, students should, within the bounds of the Honor Code, ask and answer questions on the
Slack team for our course; please request assistance from the instructor, teaching assistants, and tutors first through
Slack before sending an email. Students who need the course instructor's assistance must schedule a meeting through
\instructorpronoun{} web site and come to the meeting with all of the details needed to discuss their question.

\subsubsection*{Using Email}

Although we will primarily use Slack for class communication, I will sometimes use email to send announcements about
important matters such as changes in the schedule. It is your responsibility to check your email at least once a day and
to ensure that you can reliably send and receive emails. This class policy is based on the statement about the use of
email that appears in {\em The Compass}, the College's student handbook; please see the instructor if you do not have
this handbook.

\subsubsection*{Honor Code}

The Academic Honor Program that governs the entire academic program at Allegheny College is described in the Allegheny
Academic Bulletin. The Honor Program applies to all work that is submitted for academic credit or to meet non-credit
requirements for graduation at Allegheny College. This includes all work assigned for this class (e.g., examinations,
laboratory assignments, and the final project). All students who have enrolled in the College will work under the Honor
Program. Each student who has matriculated at the College has acknowledged the following pledge:

\vspace*{-.11in}
\begin{quote}
  I hereby recognize and pledge to fulfill my responsibilities, as defined in the Honor Code, and to maintain the
  integrity of both myself and the College community as a whole.
\end{quote}
\vspace*{-.11in}

\noindent It is understood that an important part of the learning process in any course, and particularly one in
computer science, derives from thoughtful discussions with teachers and fellow students. Such dialogue is encouraged.
However, it is necessary to distinguish carefully between the student who discusses the principles underlying a problem
with others and the student who produces assignments that are identical to, or merely variations on, someone else's
work. While it is acceptable for students in this class to discuss their programs, data sets, and reports with their
classmates, deliverables that are nearly identical to the work of others will be taken as evidence of violating the
\mbox{Honor Code}.

\subsubsection*{Disability Services}

The Americans with Disabilities Act (ADA) is a federal anti-discrimination statute that provides comprehensive civil
rights protection for persons with disabilities. Among other things, this legislation requires all students with
disabilities be guaranteed a learning environment that provides for reasonable accommodation of their disabilities.
Students with disabilities who believe they may need accommodations in this class are encouraged to contact Disability
Services at 332--2898. Disability Services is part of the Learning Commons and is located in Pelletier Library.
Please do this as soon as possible to ensure that approved accommodations are implemented in a timely fashion.

\subsection*{Welcome to an Adventure in Computer Science}

In reference to software, Frederick P.\ Brooks, Jr.\ wrote in chapter one of {\em The Mythical Man Month}, ``The magic
of myth and legend has come true in our time.'' Software is a pervasive aspect of our society that changes how we think
and act. Efficient and correct software also has the potential to positively influence the lives of many people.
Moreover, the design, implementation, testing, and evaluation of software are exciting and rewarding activities! At the
start of this class, I invite you to pursue, with great enthusiasm and vigor, this adventure in algorithms and data
structures.

\end{document}
